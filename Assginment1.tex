\documentclass[10pt]{article}
\usepackage[margin=1in]{geometry}
\usepackage{amsmath}
\usepackage{graphicx}
\usepackage{bookmark}
\usepackage{hyperref}
\usepackage{float}
\usepackage{amssymb}
\usepackage{setspace}
\usepackage{listings}
\title{Film Appreciation(Assignment 1)}
\author{V Sri Charan Reddy\\CS16BTECH11044}
\date{14/11/18}

\begin{document}
	\onehalfspacing
\maketitle
\section{Good Will Hunting(by Gus Van Sant)}
\par
Good Will Hunting is an American drama flick of 1997 which was directed by Gus Van Sant, and starring Robin Williams, BenAffleck, Matt Damon and  Minnie Driver. Affleck and Damon wrote it. It is the story of a young man's travail to find his position in the world by primarily finding out who he is. The movie is a walk through the mind of Will Hunting as he is compelled to endure therapy alternatively of jail. Robin Williams plays the role of psychologist. With the help of psychologist Williams, Will discovers himself and recognises his value in the world by understanding what matters to him most. BenAffleck plays as Will's best friend, Chuckie.
\\
\par
 Several aspects of movie techniques such as editing, camera movement can be seen in the film.During the entire movie natural colours are used. A sense of humanity and empathy for Will is created by using the warm ambers. The amber colour also signifies Will was feeling safe and comfortable in his place(in the room at Harvard, apartment or Sean's office). Warm shades indicate Will being himself during these safe situations. When he doesn't feel safe a noticeable colour shift to the cold colours can be seen. Cold tones create a sense of unrest usually. The coldness of the shot generates a feeling of thought, and we can conclude that by seeing Will lashing out when he is in an uncomfortable situation.\\
 \par
We can see a different kind of settings in the movie during the play. After the first meeting with Will, Sean sits in a dirty apartment with dishes spread across and drinking while sitting. The whiskey glass is shown as dominant signifying that he is alcoholic or that he drinks often.The closeness of the camera to characters gives the viewer closeness to them. Periodic scenes in the film describe Will going to work on a train. The High angle shots and the low angle shots were used in the film to signify the importance of power. At the end of the movie Will burns the paper with the math proof as he couldn't do it which the professor gave to him, a low angle was shot on Will signifying the power, and the professor was shot at a high angle. \\
\par
In the opening scene, dolly movement of the camera is seen when Will was reading in his apartment which gives a sense of aloneness with books as his company. In many scenes, the close-up movement of the camera can be seen which caused a sense of closeness as if we were with them. The camera is shown as an eye to create a sense of emotional connection with the characters.\\
\par
We can see that the movie begins by establishing Will and his life. We can also see that multiple pictures morphing slowly into one shot of Will alone at his apartment which is known as the kaleidoscope effect. This effect gives us an illusion that Will himself is a person with many pieces that made him a whole. Some over-the-shoulder shots can also be seen in the movie when he begins to tell the story at the National Security Council interview but gets cut to Will's face in Sean's office, saying the same story. Another instance of a cut can be seen when Skylar was waiting at the airport looking around Will to show up with a hope which gets cut to Will sitting on park bench seeing the planes leave. This scene creates the instance of empathy between Will and Skylar, knowing they want to be together, but recognising they won't be.\\
\par
In conclusion, we can say that many creative and wide range of techniques were utilised in this film for depicting a story which otherwise could be dull. The effects used in the processing and the final film causes the audience to think about the different aspects of Will which creates a sense of empathy for Will Hunting.
\end{document}
